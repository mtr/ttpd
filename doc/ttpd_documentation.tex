\documentclass[a4paper,english,11pt,twoside,openright]{book}
\usepackage{babel}
\usepackage[T1]{fontenc}
\usepackage[utf8]{inputenc}

%\usepackage{babel}
%\usepackage[lining]{minion}
%\usepackage[oldstyle]{minion}
%\usepackage[T1]{fontenc}
%\usepackage[latin1]{inputenc}
\usepackage{palatino}
\usepackage{mathpazo}
\usepackage{amsmath,amsfonts,amssymb}
\usepackage{titlesec}
\usepackage{achicago}
\usepackage{calc}
\usepackage[mathcal]{euscript}
\usepackage[leftno,noindent,norules]{lgrind}
\usepackage{layout}
\usepackage{subfigure}
\usepackage{synttree}
\usepackage{varioref}
%\usepackage{color}
\usepackage{booktabs}
\usepackage{verbatim}

\usepackage{graphics}
\usepackage[dvips]{graphicx}
\usepackage{epsfig}
\usepackage{xspace}

\usepackage{longtable}

\bibliographystyle{achicago}

% Needed because of API documentation.
\usepackage{alltt, parskip, fancyheadings, boxedminipage}
\usepackage{makeidx, multirow, longtable, tocbibind, amssymb}

\usepackage{epyfix}

\newlength{\BCL} % base class length, for base trees.

\newenvironment{Ventry}[1]%
{\begin{list}{}{%
      \renewcommand{\makelabel}[1]{\texttt{##1:}\hfil}%
      \settowidth{\labelwidth}{\texttt{#1:}}%
      \setlength{\leftmargin}{\labelsep}%
      \addtolength{\leftmargin}{\labelwidth}}}%
  {\end{list}}

\usepackage{index}

%\usepackage{concept}
\usepackage[nomarginplain,nomarginacronym,nomarginperson]{intex}
%\def\persondef#1[#2]#3{}
%% $Id: acronyms.tex,v 1.5 2002/06/03 21:26:56 mtr Exp $
%
\chapter*{Abbreviations}
\addcontentsline{toc}{chapter}{\numberline{}List of Abbreviations}

\begin{longtable}[l]{@{}ll@{}}
  %\caption{Results of first large simulation.}\\
  %\toprule
  %Acronym & Meaning \\
  %\midrule
  \endhead
  %\bottomrule
  \endfoot
  \input{alist.txt}
\end{longtable}

%\begin{multicols}{2}
%\begin{tabular}{@{}lp{.30\textwidth}@{}}
%\input{alist.txt}
%\end{tabular}
%\end{multicols}


% Local Variables:
% mode: latex
% eval: (set-default 'ispell-local-dictionary "english")
% mode: flyspell
% comment-column:0
% comment-start: "%"
% comment-end: ""
% TeX-master: "main"
% End:

%\input{persons}
%\newcommand{\coa}[1]{\emph{#1}}
%\newcommand{\cop}[1]{\emph{#1}}
%\renewcommand{\cofont}[1]{\emph{#1}}
%\renewcommand{\coafont}[1]{\emph{#1}}
%\renewcommand{\copfont}[1]{\textsl{#1}}
\usepackage[pagebackref]{hyperref}

% Indexing commands.
\def\DefaultIndex{Index\xspace}
\renewindex{default}{ddx}{dnd}{\DefaultIndex}

\def\SrcIndex{Function Index\xspace}
\newindex{src}{sdx}{snd}{\SrcIndex}

% Definition of acronyms used throughout the text.
\acrodef{API}{application program interface}
\acrodef{CPU}{central processing unit}
\acrodef{I/O}{input/output}
\acrodef{SMS}{Short Message Service}
\acrodef{TAD}{TUC Alert Daemon}
\acrodef{TTPD}{TUC Transfer Protocol Daemon}
\acrodef{TUC}{The Understanding Computer}
\acrodef{XML}{Extensible Markup Language}

\title{TUC Transfer Protocol Daemon \\ \textit{\&} \\ TUC Alert
  Daemon: \\[2em] System Documentation}

\author{Martin Thorsen Ranang}

\newcommand{\code}[1]{\texttt{#1}}


\begin{document}

\frontmatter
\maketitle

\tableofcontents
%\listoffigures
%\listoftables

%\input{preface}
%\input{acknowledgments}

% $Id: acronyms.tex,v 1.5 2002/06/03 21:26:56 mtr Exp $
%
\chapter*{Abbreviations}
\addcontentsline{toc}{chapter}{\numberline{}List of Abbreviations}

\begin{longtable}[l]{@{}ll@{}}
  %\caption{Results of first large simulation.}\\
  %\toprule
  %Acronym & Meaning \\
  %\midrule
  \endhead
  %\bottomrule
  \endfoot
  \input{alist.txt}
\end{longtable}

%\begin{multicols}{2}
%\begin{tabular}{@{}lp{.30\textwidth}@{}}
%\input{alist.txt}
%\end{tabular}
%\end{multicols}


% Local Variables:
% mode: latex
% eval: (set-default 'ispell-local-dictionary "english")
% mode: flyspell
% comment-column:0
% comment-start: "%"
% comment-end: ""
% TeX-master: "main"
% End:


%\input{abstract}


\chapter{Abstract}
\label{cha:abstract}

This document provides an overview of both the \co{TTPD} and the
\co{TAD} software.  The software package contains two programs,
named \code{ttpd} and \code{ttpc}.  These programs use a common
\co{API}, which is also documented herein.

\mainmatter

\chapter{Overview}
\label{cha:overview}

The main purpose of \co{TTPD} is to work as a mediator for requests
between external programs and \co{TUC}.  In addition, if used with
the \co{TAD} it will also handle `alert'\footnote{That is, messages
  like ``Kan du varsle meg 15 minutter før bussen fra Nardo til
  Gløshaugen går?''} requests when used in conjunction with the
\co{SMS} interface.  Figure~\ref{fig:overview} shows an overview of
the system architecture.

Both programs were implemented in Python version 2.3.4.

\begin{figure}[htbp]
  \centering
  \ifpdf%
    \includegraphics[width=1.05\textwidth]{ttpdarchitecture_1.pdf}%
  \else%
    \includegraphics[width=1.05\textwidth]{ttpdarchitecture_1.eps}%
  \fi
  \caption{Overview of the \protect\co{TTPD} and \protect\co{TAD}
    system.}

  \label{fig:overview}
\end{figure}


\section{TUC Transfer Protocol Daemon}
\label{sec:tuc-transf-prot}

The \co{TTPD} is implemented as a \co{threading} server, listening
for connections from \co{clients} on a user-specified network port (by
using a \co{socket}).  When a request is received, the server starts a
\co{handler thread} responsible for handling the request.  First, the
nature of the request is determined.  If it is an `alert cancellation'
request, the request is handled by the \co{handler thread} through
communication with \co{TAD} and the \co{client} (see
Section~\ref{sec:tuc-alert-daemon} for more information about this
behavior).  However, if the request is not of that nature, it is
forwarded to \co{TUC} for processing.  This is done by having the
server act as a \co{producer thread} that places incoming \co{tasks}
in a \co{thread pool} with $n \in \langle0, k]$ \co{consumer threads}.  Each
of the \co{consumer threads} controls \emph{one} external \co{TUC}
process each.

The external \co{TUC} processes are started by the server before it
starts accepting connections.  This is done because it would take too
long if an external \co{TUC} process should be started in order to
handle each request.  In relation to the main server process, the
\co{TUC} are \co{forked} processes.  The communication between each
\co{consumer} thread and its \co{TUC} process is done through
\co{pipes} that function as the \co{TUC} process' \code{stdin} and
\code{stdout} file streams.

When the result from \co{TUC} has been received, the \co{consumer
  thread} stores it in a \co{thread-safe} container that only the
original \co{handler thread} and the \co{consumer thread} shares.
Hence, the response for handling the request is again returned to the
\co{handler thread}.  Now, the result is parsed, the appropriate
information is logged and an answer is given to the client.

\section{TUC Alert Daemon}
\label{sec:tuc-alert-daemon}

The \co{TAD} is built-in as a part of \co{TTPD}, but has its own
responsibilities.

The most central component of \co{TAD} is a \co{scheduler}
responsible for the timing of sending out alert-messages at the
moments specified by users.

\co{TAD} consists mainly of two threads that for the most of the time
run independently of the \co{TTPD} process.  One thread constitutes
an \co{alert scheduler} while the other thread handles requests of
adding and removing alerts and is responsible for communicating with
the \co{database engine}.  The communication between a \co{TTPD
  request handler} and the \co{TAD request handler} is done through a
shared request pool (thread-safely protected by \co{lock}).



\chapter{Features}
\label{cha:features}

This chapter presents the design criteria and features that
characterizes \co{TTPD} and \co{TAD}.

\section{Design Criteria}
\label{sec:design-criteria}

The main goals of this work was to make the system more robust and to
implement an alert service.  In addition to this, where there have
been several possible solutions to a problem, the guiding principles
has been as follows: choose solutions that are \emph{easily
  maintainable} (use standard modules and programs if available, and
write easily readable code and documentation), \emph{highly scalable}
and that will \emph{probably not need to be changed} in the near
future.  In addition, the system has been designed in the spirit of
\co{Unix} philosophy; embracing modularity, the use of stream
redirection and process control as described
in~\cite{ranang03:fragrance_of_unix}.  For a very concise introduction
to the theory and pragmatics of server/daemon design, please
see~\cite{327245,327308,327345}.


\section{Scalability and Robustness}
\label{sec:scalability}

First of all, the \co{scalability} of the system has been improved in
several ways:

\begin{itemize}
\item The design of \co{TTPD} allows a computer to serve multiple
  \co{TUC} requests in parallel.  This is ensured through the use of
  the pool of \co{TUC-threads} where each thread controls an
  externally running \co{TUC} process.  This means that if more
  \co{CPU} are added to the computer, each processes can run on its
  own \co{CPU}.  Even without multiple \co{CPU}, the computer can
  serve multiple requests in multiplexed parallel (but possibly with a
  longer delay than when serving single requests, depending both on
  \co{CPU}- and \co{I/O} intensity).

\item By combining the use of \co{socket} with \co{tread} (could
  have used \co{forked processes} but they take longer to create) the
  \co{daemon} is able to stack up incoming requests and serve them as
  soon as possible.  This feature ensures that the clients can deliver
  their requests even when other requests are already being processed.

\end{itemize}

Secondly, the following features have been implemented to ensure a
high level of \co{robustness} of the system:

\begin{itemize}

\item The use of \co{socket} ensure that \emph{if} the \co{daemon}
  process should crash, any transaction received but not finished will
  silently be lost.  That is, the log will contain information about
  the reception of the requests, but when the daemon is restarted, it
  will not result in a bombardment of the users with their old
  requests.  An exception from this is the handling of any due
  \co{TAD} alerts.  They will be sent out as soon as possible after
  the alert moment have passed.


\item Efforts have been made to graciously handle extremely high
  workloads.  The daemon has a configurable ``soft'' high-load limit.
  This results in a predefined message being sent back to users---when
  the current number of concurrently handled requests is above this
  limit---telling the user that due to very high demand, the request
  cannot be handled at this moment.

\item The daemon tries to cope with non-perfect conditions in a
  controlled way:
  
  \begin{itemize}
    
  \item If one of the encapsulated \co{TUC} processes die
    unexpectedly, the daemon will start a replacement process
    immediately (this constitutes a built-in \co{watch dog} feature).

  \item All alerts that are registered by the \co{TAD} is immediately
    saved to disk.  This is done to ensure that during startup, the
    daemon consults the old log file to restore it state.

  \item Because the \co{protocol} for \co{communication} with the
    \co{SMS} \co{message switch} defines the communication as
    asynchronous, it is possible that the daemon experiences
    situations when it cannot connect to the remote server to deliver
    its answer.  If such a situation should arise---and it has, during
    intensive testing of the system---a \co{retry/resend algorithm}
    has been implemented.  The \co{handler} of that request will sleep
    for a random number of seconds (within a predefined interval) and
    then try to resend the answer when it awakens.  If it still does
    not succeed, it repeats this behavior until a total time limit has
    been reached (suggestively sat to 15 minutes).

  \end{itemize}

\item Since the \co{SMS message switch} \co{communication protocol}
  uses \co{XML}-based headers, the daemon parses these with the
  default \co{Python} \co{SAX-parser} (usually an \co{expat parser}).
  This ensures a robust handling of incoming messages and an automatic
  check for \co{XML} compliance.

\item The \co{TTPD} is designed to handle random connections to its
  socket, so that non-protocol clients will not crash it.

\end{itemize}


\chapter{Program Usage}
\label{cha:program-usage}

The programs \code{ttpd} (the \co{daemon}) and \code{ttpc} (the client
interface application) both contain some built-in help information
that is always available from the command line.  All you need to do is
to supply one of the flags \code{--help} or \code{-h} to the program
on the command line.

\section{TUC Transfer Protocol Daemon}
\label{sec:tuc-transf-prot-1}

The help menu in \code{ttpd} looks like:

{\small
\verbatiminput{usage_ttpd.txt}
}

\section{TUC Transfer Protocol Daemon Controller}
\label{sec:tuc-transf-prot-3}

The help menu in \code{ttpdctl} looks like:

{\small
\verbatiminput{usage_ttpdctl.txt}
}

\section{TUC Transfer Protocol Client Application}
\label{sec:tuc-transf-prot-2}

The help menu in \code{ttpc} looks like:

{\small
\verbatiminput{usage_ttpc.txt}
}


%\input{introduction}
%\input{methods}
%\input{results}
%\input{discussion}

\appendix

\chapter{Module Application Program Interface}

It should be noted that the SocketServer module was not developed by
the author, but since the modules TTP.Server and TTP.Handler contain
classes that inherit from classes in SocketServer, it has been
included here to make the \co{API} documentation meaningful.

\section{ttpd}\lgrindfile{doc_src/ttpd.tex}
\section{TUCThread.py}\lgrindfile{doc_src/TUCThread.py.tex}
\section{EncapsulateTUC.py}\lgrindfile{doc_src/EncapsulateTUC.py.tex}
\section{TTPDLogHandler.py}\lgrindfile{doc_src/TTPDLogHandler.py.tex}
\section{num\_hash.py}\lgrindfile{doc_src/num_hash.py.tex}
\section{tad.py}\lgrindfile{doc_src/tad.py.tex}
\section{tad\_test.py}\lgrindfile{doc_src/tad_test.py.tex}
\section{ThreadPool.py}\lgrindfile{doc_src/ThreadPool.py.tex}
\section{ttpd\_client\_test.py}\lgrindfile{doc_src/ttpd_client_test.py.tex}


% \chapter{Source Code}

% {
% % A little hack to create a ``function index''.
% \makeatletter%
% \let\oldindex\index%
% \newcommand{\mtrindex}[1]{%
% \oldindex[src]{#1}%
% }
% \let\index\mtrindex%
% \makeatother%
% \input{include_src_src} % Include source-code listings.
% \let\index\oldindex%
% }

%\section{ttpd}\lgrindfile{doc_src/ttpd.tex}
\section{TUCThread.py}\lgrindfile{doc_src/TUCThread.py.tex}
\section{EncapsulateTUC.py}\lgrindfile{doc_src/EncapsulateTUC.py.tex}
\section{TTPDLogHandler.py}\lgrindfile{doc_src/TTPDLogHandler.py.tex}
\section{num\_hash.py}\lgrindfile{doc_src/num_hash.py.tex}
\section{tad.py}\lgrindfile{doc_src/tad.py.tex}
\section{tad\_test.py}\lgrindfile{doc_src/tad_test.py.tex}
\section{ThreadPool.py}\lgrindfile{doc_src/ThreadPool.py.tex}
\section{ttpd\_client\_test.py}\lgrindfile{doc_src/ttpd_client_test.py.tex}


%\input{ideas}
%\input{non_public}

%\nocite{*}

\bibliography{bibliography}


%% The function index.
\newpage{\thispagestyle{empty}\cleardoublepage}
\addcontentsline{toc}{chapter}{\numberline{}\SrcIndex}
\printindex[src]

%% The main index.
\newpage{\thispagestyle{empty}\cleardoublepage}
\addcontentsline{toc}{chapter}{\numberline{}\DefaultIndex}
\printindex[default]

\end{document}

%%% Local Variables: 
%%% mode: latex
%%% buffer-file-coding-system: utf-8-unix
%%% TeX-master: "ttpd_documentation"
%%% mode: flyspell
%%% End: 
